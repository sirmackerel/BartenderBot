
% Andrew G. West - prop_spec.tex
% Main LaTeX file for CIS400/401 Project Proposal Specification
%
% Once built and in PDF form this document outlines the format of a project proposal. However, in raw (.tex) form, we also try to comment on some basic LaTeX technique. This is not intended to be a LaTeX tutorial, instead just (1) a use-case thereof, and (2) a template for your own writing.

	% AW - Ordinarily we'd begin by specifying some broad document properties like font-size, page-size, margins, etc. -- We have done this (and much more) for you by creating a 'style file', which the 'documentclass' command references.
\documentclass{sig-alternate}
 
	% AW - These 'usepackage' commands are a way of importing additional LaTeX styles and formattings that aren't part of the 'standard library'
\usepackage{mdwlist}
\usepackage{url}

\begin{document} 

	% AW - We setup the parameters to our title header before 'making' it.
\title{CIS400/401 Project Proposal Specification - BartenderBot}
\subtitle{Dept. of CIS - Senior Design 2010-2011}
\numberofauthors{3}
\author{
\alignauthor Seth Shannin \\ \email{sshannin@seas} \\ Univ. of Pennsylvania \\ Philadelphia, PA
\alignauthor Kevin Xu \\ \email{kevinxu@seas} \\ Univ. of Pennsylvania \\ Philadelphia, PA
\alignauthor Camillo Jose Taylor \\ \email{cjtaylor@cis} \\ Univ. of Pennsylvania \\ Philadelphia, PA}
\date{}
\maketitle

	% AW - Next we write out our abstract -- generally a two paragraph maximum, executive summary of the motivation and contributions of the work.
\begin{abstract}
\textit{Programming a robot to perform human tasks has been the focus of many research papers. The task has been traditionally very challenging, as it involves heavy computation and complicated coordination between many different joints. Here we present a method to teach the Willow Garage PR2 robot how to mix and serve a drink through imitation. By using immersive teleoperation, it is possible to issue complex commands to the PR2 and have it shadow human motion. The teleoperation will be provided by the Microsoft XBox Kinect, and communication with the PR2 will be handled with ROS. The goal of this project is to have the PR2 not only shadow human motion captured by the Kinect, but also learn from this motion and adapt to new situations via reinforcement learning techniques. We will outline our plan for achieving both shadowing and learning with the Kinect, PR2, and ROS.}
\end{abstract}

	% AW - Then we proceed into the body of the report itself. The effect of the 'section' command is obvious, but also notice 'label'. Its good practice to label every (sub)-section, graph, equation etc. -- this gives us a way to dynamically reference it later in the text via the 'ref' command.
\section{Introduction}
\label{sec:intro}
Constructing a fully autonomous and adaptive robot has been a long-time goal of robotics research. There have been many different attempts at overcoming the challenges involved in developing such a robot. The ability to learn is a powerful intermediate step towards full autonomy. A common problem involves choosing how exactly to demonstrate a desired behavior in such a way that a robot can learn that behavior. In this paper we propose to teach a WillowGarage PR2 Robot how to perform a reasonably complex task (mixing a drink) through shadowing of human motion capured by the Microsoft XBOX Kinect. The XBOX Kinect sensor from Microsoft provides real-time depth information from a scene at 30 FPS. Combined with various open source libraries\footnote{http://openkinect.org/wiki/Main\_page}, the Kinect has been used in many projects involving real-time tracking of human motion, many of which can be found online\footnote{http://www.freenect.com/}. The PR2 is a humanoid robot developed by  Willow Garage\footnote{http://www.willowgarage.com/pages/pr2/overview} for the purpose of robotics research. It is has been taught by different teams to do many things, including baking cookies\footnote{http://spectrum.ieee.org/automaton/robotics/home-\\robots/pr2-learning-to-bake-cookies-humanity-\\surrenders-to-yumminess}, scanning and bagging groceries\footnote{http://spectrum.ieee.org/automaton/robotics/robotics-\\software/pr2-can-scan-and-bag-your-groceries}, and fetching a sandwich from Subway\footnote{http://spectrum.ieee.org/automaton/robotics/artificial-\\intelligence/pr2-can-now-fetch-you-a-sandwich-from-\\subway}.

We propose to use the Kinect to map human movement to teach the PR2 how to mix and pour a drink. The Kinect sensor provides a convenient way to demonstrate the desired behavior by tracking human motion. The captured data can be relayed to the PR2 via ROS, an open-source Robot Operating System\cite{ros}. ROS provides a convenient framework for inter-process communication and coordination between different sensors and components of the PR2. ROS enables relatively short programs to issue surprisingly sophisticated commands to the PR2, such as continually tracking a moving point over time\cite{ros_pr2}. By using ROS, Kinect, and the PR2, we will demonstrate the effectiveness of teleimmersive demonstration learning in teaching a robot new behavior. This method has several advantages over existing approaches. First of all, the Kinect sensor provides accurate real-time human motion tracking that can be translated to joint movement in the PR2 thanks to ROS. Secondly, teleimmersion better enables a human teacher to show a robot learner exactly how to move in a given situation compared to kinesthetic learning, which involves manipulating the robot learner directly by physical contact. Teleimmersion will also allow demonstrations for robots that cannot be subject to kinesthetic learning easily, such as very large or very small robots. Our method, if successful, would allow for rapid introduction of all kinds of different behavior to the PR2 all from human motion. This technique could be generalized to other humanoid robots besides the PR2 to teach them different behavior.
	% AW - The header of this document might have been a little intimidatating to beginners. Notice once you are in the body of the document, however, LaTeX commands are minimal and 'normal text' is frequent.
\section{Related Work}
\label{sec:related_work}
Perhaps the most important section of your proposal is \textit{related work}. Here you demonstrate that you have read and understand what others in the field have done. This ensures you (1) know the state-of-the-art, (2) are not re-doing others work, and (3) you know the performance levels you must achieve to make a contribution. As you discuss each related work, make note of how each has advanced the field. More importantly, note their shortcomings -- these you will exploit to come up with a \textit{better} system. 

	% Here we see our first citations. It's just a simple command, the body of which is the keyword-label assigned to resources over in the *.bib file
This section should have in-line citations to your bibliography (really all sections should have citations, but we expect them to be most dense in this section). We are going to require that your proposal has at least $6$ references. Fortunately, \LaTeX{} makes citations easy. Your two TAs have had no difficulty, as the work of West \textit{et al.}~\cite{quantm} demonstrates. Need help with \LaTeX{}? Be sure to check out~\cite{latex_wikibook} and~\cite{ctan_pdf}, two helpful on-line resources.

What defines a good resource? Wikipedia is \textbf{NOT} a good resource. We would like to see references from academic journals/conferences (ACM, IEEE, etc.). We realize not everyone is doing pure research and for students with `implementation' projects such sources may be rare. No matter the case, your sources need to be reputable. You will later be asked to produce an \textit{annotated bibliography} where you defend the quality of your resources.

Let us return to your factorization proposal. You should put out the earliest related work; na\"{i}ve methods like trial divison and the Sieve of Eratosthenes, but state they are of no modern relevance. Then discuss modern methods like the Quadratic Sieve\footnote{This was the topic of one TAs undergraduate thesis, and the only reason this particular example was chosen.} and General Number Field Sieve. Note the humongous time and memory bounds of these algorithms. But wait! You are going to propose a better way $\ldots$

There have been many other projects involving autonomous robots and handling drinks. Hillenbrand \textit{et al.}~\cite{pouring_arm} designed a semi-autonomous hand-arm robot for serving drinks. The robot was capable of responding to user input by choosing a drink from multiple choices, opening it, and pouring it into a glass, and then offering the drink to the user. The hand was capable of not only picking up bottles and cups, but also unscrewing bottle caps. The robot combined stereo processing and object recognition to identify the drinks, and then used grasp planning to pick up the drink itself. Bohren \textit{et al.}~\cite{beer} used the PR2 and ROS to build a robotic system for retrieving a beer from a refrigerator. In their work, they developed a task-level execution system known as SMACH for rapidly prototyping robotic applications. The PR2 had to navigate an obstacle map to reach the refrigerator, use object recognization and grasp planning to identify the door handle and the drinks, and ultimately use facial recognition to deliver the beer to a human recipient. Each step of the process contained detail planning and image processing in order to carry out the expected behavior. 

\section{Project Proposal}
\label{sec:project_proposal}
Now is the time to introduce your proposed project in all of its glory. Admittedly, this is not the easiest since you probably have not done much actual research yet. Even so, setting and realizing realistic research goals is an important skill. Begin by summarizing what you are going to do and the expected benefit it will bring.

\subsection{Anticipated Approach}
\label{subsec:approach}
Having summarized \textit{what} you are going to do, its time to describe \textit{how} you plan to do it. Our factorization example does not work so well here (its likely impossible to realize) -- so let us suppose you are going to create a service that takes a cell-phone picture of a building and returns via text-message, the name of that building\footnote{Do not use this idea -- someone did it last year.}. 

In this case you might want to talk about establishing a server to receive pictures via MMS. Once the picture is received, you will run an edge extraction algorithm over it. Then, similarity between the submitted picture and those stored (and tagged) in a MySQL database will be computing using algorithm $XYZ$. Finally, the tag of the most similar image will be returned to the user. Do not bore the reader with trivial details, but give them an overview; a block-flow diagram may prove extremely helpful.

Also note where you anticipate having \underline{novel} difficulty. Maybe you have never setup a MySQL database or even used SQL before at all -- yes, that is a challenge -- but not one reader's care about. More novel would be the fact that many buildings on Penn's campus look similar and your classifier may be inaccurate in such instances.

\subsection{Evaluation Criteria}
\label{subsec:eval_criteria}
Suppose you have implemented your approach, and it is functioning. Now how are you going to convince readers your approach is better than what exists? In the factorization example, you could just compare run-times between algorithms run on the same input. The image recognition example might use a percentage of accurate classifications. Other fields may have established testing benchmarks.

No matter the case, you need to prove you have contributed to the field. This will be easier for some than others. In particular, those with `sensory' projects involving visual or sonic elements need to think this point through -- objectivity measures are always better than subjective ones.

\section{Research Timeline}
\label{sec:research_timeline}
Finally, we would like you to speculate about the pace of your research progress. This section need not be lengthy, we would just like you to specify some milestones so we can gauge your progress during our intermediate interviews. Let us follow through with our image recognition example:

	% The 'itemize' environment shown here, and its friend 'enumerate' (shown below), are used to create indented\bulleted\outline style lists.
\begin{itemize*}
	\item {\sc already completed}: Preliminary reading. Began implementation of image-recognition algorithm.\vspace{3pt}
	\item {\sc prior-to thanksgiving} : Photograph buildings for DB. Make algorithm more efficient, tune parameters.\vspace{3pt}
	\item {\sc prior-to christmas} : Create server-MMS interface. Expand tagged DB collection.\vspace{3pt}
	\item {\sc completion tasks} : Verify implementation is bug-free. Conduct accuracy testing. Complete write-up.\vspace{3pt}
	\item {\sc if there's time} : Investigate image pre-processing techniques to improve accuracy.
\end{itemize*}

	% AW: We next move onto the bibliography.
\bibliographystyle{plain} % Please do not change the bib-style
\bibliography{prop_spec}  % Just the *.BIB filename

	% AW: Here is a dirty hack. We insert so much vertical space that the appendices, which want to begin in the left colum underneath "references", are pushed over to the right-hand column. If we looked hard enough, there is probably a command to do exactly this (and wouldn't need tweaked after edits).
\vspace{150pt}

	% AW: We then use appendices to share some additional information with you, though you won't need appendices in your own proposal.
\appendix
\section{Other Specifics}
\label{app:other_specifics}
Your proposal need not have appendices like this section and the next, but we still have critical info to share:

	% The usage of 'enumerate' (similar to 'itemize') we talked about above
	%
	% You may also notice we have many 'vspace' commands lying around. These create 'vertical space' and are a way to force LaTeX to cooperate, sometimes. Don't get too involved with using them initially, though, because adding or deleting a single line of task can dramatically change how LaTeX chooses to format, page, and space the document
\begin{enumerate*}
	\item {\sc proposal length}: We require that your proposal be 4--5 pages in length, bibliography included. Be careful, \LaTeX{} and our style-file in particular are \textit{extremely} space efficient. An 9-page MS-Word document could easily become a 5-page \LaTeX{} one.\vspace{5pt}
	\item {\sc plagarism}: \textbf{DO NOT} plagarize. If you are caught, you will fail the class (\textit{i.e.}, not graduate), or worse. 
\end{enumerate*}

\section{\LaTeX{} Examples}
\label{app:latex_examples}

	% This paragraph makes use of dynamic references. Remember how we've been 'label'-ing everything; sections, etc? Using 'ref' we can reference them. Add a new figure/section at the beginning? This technique automatically re-numbers when you build, so you don't have to make static changes.
At this point, the proposal specification is complete. From here on out, we are just going to show off some commonly used \LaTeX{} technique. Be sure to look at the `code behind' and see Tab.~\ref{tab:some_table}, Eqn.~\ref{eqn:some_equation} and Fig.~\ref{fig:some_graph} for the output!

	% AW - Math is obviously one of LaTeX's strengths. Math can be typeset in-line, or off-set in equation 'environments' like this. You'll need to look up symbols on an as-needed basis, but I'll assure you -- they are ALL there.
\begin{equation}
M(p) = \int^\infty_0 (1+\alpha x)^{-\gamma}x^{p-1}dx
\label{eqn:some_equation}
\end{equation}

	% AW - We next encounter tables and figures (images). Big things like these are known as 'floats' in LaTeX because their position is not fixed. Notice that '[htb!]' follows the start of each environment. We are telling LaTeX that we'd like to put the table/fig 'h' - HERE, precisely where it follows in the narrative. If LaTeX determines it doesn't look good here, 't' tells LaTeX we'd like it at the top of this column, and if that doesn't work, use 'b', the bottom of the column. Other options are available. LaTeX shifts floats around to ensure images don't end up on page/column boundaries, which would result in a waste of space for text.

	% AW - We insert a table into the document. Notice the '| c | c | c |' argument provided to the tabular environment. This says we want three columns, all with center-alignment, with vertical bars between them. In the body of argument, an ampersand '&' separates cells, and a double forward-slash '\\' is used to create new lines. Otherwise, commands should be self explanatory.
\begin{table}[htb!]
	\begin{center}
  \begin{tabular}{| c | c | c |}
    \hline
    \textbf{User Type} & \textbf{Cleanup\%} & \textbf{Honesty\%} \\ \hline
    Good & 90-100\% & 100\% \\ \hline
    Purely Malicous & 0-10\% & 0\% \\ \hline
		Malicious Provider & 0-10\% & 100\% \\ \hline
		Feedback Malicous & 90-100\% & 0\% \\ \hline
		Disguised Malicous & 50-100\% & 50-100\% \\ \hline
		Sybil Attacker & 0-10\% & Irrelevant \\ \hline
  \end{tabular}
	\caption{Example Table}
  \label{tab:some_table}
	\vspace{-10pt}
	\end{center}
\end{table}

	% AW - We insert a graph/figure into the document. This is a pretty straightforward process once you get the image into a file format that LaTeX plays nice with. Then we just scale it as a % of the column width.
\begin{figure}[htb!]
	\begin{center}
		\includegraphics[width=0.75\linewidth]{some_graph}
	\end{center}
	\vspace{-12pt}
	\caption{Example Figure/Graph}
	\label{fig:some_graph}
\end{figure}

\end{document} 

